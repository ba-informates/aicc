\documentclass{article}
\usepackage{graphicx} % Required for inserting images
\graphicspath{{Images/}}
\usepackage[utf8]{inputenc}
\usepackage{multicol}
\usepackage{amsthm}
\usepackage{amsmath}
\usepackage{xcolor}
\usepackage{wrapfig}

\newtheorem{definition}{Definition}[section]
\newtheorem{remark}{Remarque}[section]
\newtheorem{theorem}{Théorème}[section]

\title{AICC}
\author{Laura Paraboschi / Simon Lefort}
\date{BA1-IN}

\begin{document}

\maketitle

\section{Types of proofs}

\subsection{Direct Proof}

e.g. Prove "if a and b are real and positive integers, then $ (a+b)^{n} \geq a^{n} + b^{n} $." for 1. \\ \\
\[ (a+b)^1 = a^1 + b^1 = a + b \]

\subsection{Proof by contrapositive}

e.g. prove P(n) : "if n is an integer and $ n^{3} + 5 $ is odd, then $ n $ is even". \\ \\
Let's assume $ n $ is odd $ n = 2k $, then (after performing a rewrite), $ n^3 + 5 = 2(...) $, and is even.

\[ p \implies q \]
\[ \equiv \neg{q} \implies \neg{p} \]

\subsection{Proof by contradiction}

e.g prove P(n). \\ \\
Let's assume $ n^3 + 5 $ is odd and $ n $ is odd. Then, $ n^2 $ is odd, $ n^3 $ is also odd, but the difference between $ n^3 $ (odd) and $ 5 $ (odd) is necessarily even.

\[ \text{let } p \text{ and } \neg{q} \]
\[ \text{show that they can not exist together.} \]

\subsection{Vacuous Proof}

e.g. prove "if n is a positive integer greater than one, then $n_{2} > n $" for 0.

\[ p \implies q \]
\[ F \implies q \]
\[ \equiv T \text{ , no matter what } q \text{ is} \]

\subsection{Trivial Proof}

same as vacuous, but we know $ q $ and not $ p $.

\[ p \implies q \]
\[ p \implies T \]
\[ \equiv T \text{ , no matter what } p \text{ is} \]

\subsection{Proof by cases}

\textcolor{red}{la difficulté est d'apprendre à choisir les cas à traiter} \\ \\
e.g prove "100 is not the cube of a positive integer". \\ \\
Poser d'abord que $ (n+1)^3 > n^3 $. \\ \\
\textbf{Cas $ 1 \leq x \leq 4 $ :} \\
$ 4^3 = 64 < 100 $ \\ \\
\textbf{Cas $ x > 4 $ :} \\
$ 5^3 = 125 > 100 $ \\ \\
$ \implies \neg{\exists{x}} (x^3 = 100) $

\end{document}
